\section{Using the Visualiser and views}
If this is the first time you use the \cjdt, switch to the \caesarj ~perspective by selecting \markedtext{Window} $\rightarrow$ \markedtext{Open Perspective} $\rightarrow$ \markedtext{Other}. Pick \markedtext{CaesarJDT Perspective} (see figure \ref{fig:select_persp}) in the list.

\begin{figure*}[htbp]
	\centering
		\includegraphics[width=0.35\textwidth]{images/select_persp.png}
	\caption{Perspective selection}
	\label{fig:select_persp}
\end{figure*}

This perspective extends the Java perspective. Especially a new view is available. The \markedtext{\caesarj ~Hierarchy View}. See section \ref{hierarchyview} for detailed information.\\
You can switch between the Java and Caesar Visualization perspectives using the perspective icons in the top right of the menu bar.\\
\subsection{Outline view}
The outline view is showing structural members and crosscutting relationships. It extends the Java outline view by additional information (e.g advice declarations to the places it advises). A sample outline view bar is shown in figure \ref{fig:outline_view}. \textbf{TODO Bild noch nicht das richtige.}\\

\begin{figure*}[htbp]
	\centering
		\includegraphics[width=0.35\textwidth]{images/outline.png}
	\caption{Outline View}
	\label{fig:outline_view}
\end{figure*}

\subsection{Hierarchy View\label{hierarchyview}}
% It displays the hierarchical relationships of \caesarj ~cclasses.
A \caesarj ~hierarchy view displays the hierarchical relationships of \caesarj ~cclasses. That means, that for each cclass their super-classes are displayed under the \markedtext{Super} node (see figure \ref{fig:hierarchy_view}). If the class contains nested classes (\markedtext{Contains} node) there are two displaying modes available for them:
\begin{itemize}
	\item[\textbf{Super:}] For each nested class their super classes are displayed.
	\item[\textbf{Sub:}] For each nested class their sub classes are displayed. If a sub class has two super classes the linearized inheritance relation is displayed in brackets after the class name.
\end{itemize}

\begin{figure*}[htbp]
	\centering
		\includegraphics[width=0.35\textwidth]{images/hierarchy.png}
	\caption{\caesarj ~hierarchy view}
	\label{fig:hierarchy_view}
\end{figure*}

The modes can be switched by pressing the control button in the upper-right of the view. The second part of the view, named "Mixin view", shows the mixin composition of the currently selected (nested-) cclass.\\
\textbf{Note:} Because this view needs meta information from the compiler, the view refreshes when a project was (re-)built successfully.

