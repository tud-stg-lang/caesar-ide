\documentclass[a4paper,12pt,twoside,titlepage]{article}
\usepackage[pdftitle={Userguide Caesar Development Tools},pdfborder=0,pdfauthor={Jochen Unger, Daniel Zwicker},colorlinks=true,linkcolor=blue,linkbordercolor=255 255 255]{hyperref}
\usepackage{amssymb}
\usepackage{graphicx}
\usepackage[ansinew]{inputenc}
\usepackage{a4}
\usepackage{listings}
\usepackage{color}
\pagestyle{headings}
\usepackage{float}
\restylefloat{figure}
\usepackage{longtable}

\title{\textbf{Userguide Caesar Development Tools}\\\small{Software Technology Group}\\\small{Version 0.1}}
\author{Jochen Unger\and Daniel Zwicker}
%Hier beginnt das Dokument

\newcommand{\code}[1]
{\textbf{"#1"}}

\newcommand{\caesarj}
{\caesar}

\newcommand{\markedtext}[1]
{\fbox{\textbf{#1}}}

\newcommand{\caesar}
{C{\footnotesize AESAR}J}

\newcommand{\jdt}
{Java Development Tool}

\newcommand{\cjdt}
{CaesarJ Development Tool}

\begin{document}
\lstset{
   frame=l,
   language=Java,
   xleftmargin=10pt, 
   stepnumber=1,
   numbers=left,
   numbersep=5pt,
   numberstyle=\ttfamily\tiny\color[gray]{0.3},
   belowcaptionskip=\bigskipamount,
   escapeinside={*'}{'*},
   tabsize=1,
   emphstyle={\it},
   commentstyle=\rmfamily\it,
   stringstyle=\mdseries\rmfamily,
   showspaces=false,
   keywordstyle=\bfseries,
   columns=flexible,
   basicstyle=\small,
   showstringspaces=false,
   framextopmargin=2pt,
   framexbottommargin=2pt
  }

\maketitle
\tableofcontents

\newpage
\section{Introduction}
This documentation discribes how to use the \caesarj -Eclipse Plugin for the new programming language \caesarj.
\subsection{What is \caesarj ?}
\caesar ~is a new aspect-oriented programming language that extends JAVA with the following functions:
\begin{itemize}
	\item aspect\footnote{\href{http://www.st.informatik.tu-darmstadt.de/database/publications/data/aosd03.pdf?id=70}{This paper} contains more information about \caesarj ~and aspects.} functionality with runtime deployment of aspects
	\item multiple inheritance
	\item aspecture polymorphisms
	\item dynamic deployment
	\item virtual classes with propagating mixin composition
	\item \dots
\end{itemize}
For more detailed information please visite \href{http://caesarj.org/}{http://caesarj.org/}.

\subsection{What is the use of the \caesarj ~ Plugin?}
\caesarj ~extends the Java source code. A pure JAVA-Editor (e.g. the \jdt -Eclipse Plugin) would not be able to handle this kind of source code. Furthermore the integrated JAVA compiler would not work. So an IDE is needed, that extends the \jdt. 
\subsection{What are the features of the \jdt}
The \jdt -Eclipse plugin supports many features for JAVA programmers. Some highlights are:
\begin{itemize}
	\item code highlighting
	\item live code annotation 
	\item outline visualisation
	\item type hierarchy
	\item calling hierarchy
	\item and many more
\end{itemize}
The \caesarj ~plugin trys to extend these features and add \caesarj ~specific features.

\subsection{Which extensions do we need for \caesarj}
To provide a good IDE for the \caesarj ~programming language, we needed to extend the \jdt ~plugin with the following features:

\begin{itemize}
	\item syntax-highlighting for \caesarj ~expressions
	\item crosscutting views
	\item integrated \caesarj ~builder
	\item new outline bar (with visualisation of aspects)
	\item new side ruler (with visualisation of aspects)
	\item \caesarj ~hierarchy view (visualisation of multiple inheritance)
\end{itemize}
For detailed description please see Section \ref{features}.




\newpage
\section{Getting Started with the \cjdt}
This section describes how to get started with the \cjdt -Plugin for Eclipse. It provides a rich set of features for working with \caesarj ~programs inside Eclipse.\\\\
\subsection{\cjdt ~Highlights:}
\begin{itemize}
	\item Editor support with keyword highlighting. (Figure \ref{fig:hilight})
	
\begin{figure*}[htbp]
	\centering
		\includegraphics[width=0.60\textwidth]{images/hilight.png}
	\caption{Codehighlighting in \cjdt}
	\label{fig:hilight}
\end{figure*}

  \item Outline view showing structural members and crosscutting relationships. Also from an advice declaration to the places it advises. (Figure \ref{fig:outline})\\ \textbf{TODO new picture}

\begin{figure*}[htbp]
	\centering
		\includegraphics[width=0.35\textwidth]{images/outline.png}
	\caption{Outline view with advice relations}
	\label{fig:outline}
\end{figure*}

  \item New \caesarj -project wizard. This wizard helps you to start a new \caesarj -project. (Figure \ref{fig:projectwizard})
  
\begin{figure*}[htbp]
	\centering
		\includegraphics[width=0.60\textwidth]{images/project_wizard.png}
	\caption{New \caesarj -project wizard}
	\label{fig:projectwizard}
\end{figure*} 
     
  \item \caesarj ~hierarchy view. This view shows the multiple inheritance and nested class relations of an \caesarj ~top level class. (Figure \ref{fig:hierarchy})
  
\begin{figure*}[htbp]
	\centering
		\includegraphics[width=0.35\textwidth]{images/hierarchy.png}
	\caption{\caesarj ~hierarchy view}
	\label{fig:hierarchy}
\end{figure*} 

  \item Debugging support. (Figure \ref{fig:debug1})
  
\begin{figure*}[htbp]
	\centering
		\includegraphics[width=1.0\textwidth]{images/debug1.png}
	\caption{Debugging an \caesarj -project}
	\label{fig:debug1}
\end{figure*}
\end{itemize}

\newpage
\section{\cjdt ~Installation}
The following two sections describe the installation of the \caesarj ~eclipse plugin. Two scenarios are possible: clean installation and updating an existing installation.
\subsection{Clean Installation} 

The \cjdt ~is installed using the Eclipse Update Manager. We recommend you to use Eclipse 3.x.
\subsubsection{Using A Proxy Server} 
If you need to use a proxy server to access the internet, the first thing
to do is to configure the proxy preference details, so that the update manager can contact the
\cjdt ~update site. From the \markedtext{Window} menu select \markedtext{Preferences} and then the
\markedtext{Install/Update} tab. Please enter your proxy server details as shown in figure \ref{fig:proxy}.
\begin{figure*}[htbp]
	\centering
		\includegraphics[width=0.60\textwidth]{./images/proxy.png}
	\caption{Setting up your proxy server}
	\label{fig:proxy}
\end{figure*}

\subsubsection{Installing via Update Manager} 
Create an update site bookmark for the \cjdt ~update site, and start the install procedure.
% Ob man das brauch? \label{sec:In Eclipse 3.0}
From the help menu, select \markedtext{Software Updates} $\rightarrow$ \markedtext{Find and Install}. Select \markedtext{Search for new features to install} and click \markedtext{Next}.
Click \markedtext{Add Update Site} and enter the name \markedtext{\caesarj ~update site} and the folowing URL:  
\begin{center}
\href{http://cage.st.informatik.tu-darmstadt.de/caesar/updatesite/0.3.1}{http://cage.st.informatik.tu-darmstadt.de/caesar/updatesite/0.3.1}
\end{center}
Click \markedtext{OK}.
Fully expand the appearing \cjdt ~update site node and select \markedtext{\caesarj}. Click
\markedtext{Next}. Select \markedtext{org.caesarj.feature} as shown in figure \ref{fig:installpage30} and click \markedtext{Next}.\\

\begin{figure*}[htbp]
	\centering
		\includegraphics[width=0.60\textwidth]{./images/install_page_3_0.png}
	\caption{Selection of the \caesarj -plugin}
	\label{fig:installpage30}
\end{figure*}

Accept the \markedtext{license agreement} and proceed to the installation.

\subsection{Updating an Existing Installation}
Proceed as for a clean install, except that the \cjdt ~update site bookmark should already
exist. All you need to do, is to expand the bookmark node and go on. If the version you have
installed is the same as the version on the update site (or more recent even),
then you will not be presented with any installing options.

\subsection{Is Everything OK?} 
Restart the Eclipse workbench. Try to open a new perspective by clicking \markedtext{Window} $\rightarrow$ \markedtext{Open Perspective}. Pick \markedtext{other} and select \markedtext{CaesarJ Perspective} in the upcoming list.
When the perspective opens successfully, the installation of your \cjdt ~works fine.
%\begin{figure}[htbp]
%	\centering
%		\includegraphics[width=0.95\textwidth]{./images/welcome_3_0.png}
%	\label{fig:welcome_3_0}
%\end{figure}



\newpage
\section{Features\label{features}}
The following section describes the extending features of the CaesarJ Plugin. 

\subsection{Opening the \caesarj -perspective}
First of all you need to open the \caesarj -perspective. It includes some new features like the \caesar -editor, the new outline view or the \caesarj -hierarchy view.\\
You can open this perspective by selecting: \markedtext{Window} $\rightarrow$ \markedtext{Open Perspective} $\rightarrow$ \markedtext{other} $\rightarrow$ \markedtext{CaesarJ perspective}.\\
If this is the first time you used the plugin, you will see the following dialog popup shown in figure \ref{fig:view_properties}.

\begin{figure}[htbp]
	\centering
		\includegraphics[width=0.5\textwidth]{images/view_properties.png}
	\caption{The \caesarj ~Preferences}	
	\label{fig:view_properties}
\end{figure}

This dialog configures some Eclipse settings that will make your life much easier when
working with \caesarj -projects. Leave everything selected and click
\markedtext{Finish}.

\subsection{Creating a new \caesarj project \label{creating_project}}
From the File menu select \markedtext{New} $\rightarrow$ \markedtext{Project}. Pick \markedtext{Caesar Project} in the list and select \markedtext{Next} as shown in figure \ref{fig:project_wizard}.

\begin{figure}[htbp]
	\centering
		\includegraphics[width=0.60\textwidth]{images/project_wizard.png}
	\caption{Coosing the New Project Wizard}
	\label{fig:project_wizard}
\end{figure}

If it doesn't appear in the list, this is probably the first time you use the plugin. Select \markedtext{Other} and then \markedtext{Caesar} and \markedtext{Caesar Project}.\\
The wizard is opened. Here specify a name for your project as shown in figure \ref{fig:project_wizard2}.

\begin{figure}[htbp]
	\centering
		\includegraphics[width=0.60\textwidth]{images/project_wizard2.png}
	\caption{The New Project Wizard}
	\label{fig:project_wizard2}
\end{figure}

This wizard has identical behavior to the new Java project wizard (except of course that
it creates a project with the Caesar nature).\\
When you click \markedtext{Finish}, your project will be created.

\subsection{Adding a Class to Your Project}
First you have to create a package for your class files. Select the project you created in the section \ref{creating_project} in the package explorer. Right click on it and select \markedtext{New} $\rightarrow$ \markedtext{Other} from the context menu. You have to look for \markedtext{Package} in the \markedtext{Java} subsection like you can see in figure \ref{fig:package}.

\begin{figure}[htbp]
	\centering
		\includegraphics[width=0.60\textwidth]{images/package.png}
	\caption{Creating a package}
	\label{fig:package}
\end{figure}

Name the package \code{myPackage} then click \markedtext{Finish}.\\
Right-click on the package you just created and select \markedtext{New} $\rightarrow$ \markedtext{Class} from the context menu. Name the class \code{HelloWorld} and activate the option to let Eclipse create a new main method for you. Click \markedtext{Finish}.
Edit the text in the editor so that it looks something like this:
%\begin{table*}[htbp]
	\begin{lstlisting}[basicstyle=\small\it,caption=HelloWorld.java,label=lst:HelloWorld,name=listing:helloworld,frame=none]{}
package myPackage;

public class HelloWorld {

	public static void main(String[] args) {
		HelloWorld hello = new HelloWorld();
		hello.sayHello("Hello");
	}

	public void sayHello(String arg) {
		System.out.println(arg);
	}
}
\end{lstlisting}
%	\caption{HelloWorld.java}
%	\label{lst:HelloWorldJava}
%\end{table*}
Save the file.\\
Notice that unlike in a Java project, there was no eager parsing of the buffer as you typed. Also the outline view didn't update. Your Eclipse workbench should be looking something like in figure \ref{fig:workspace_newclass}.\\

\begin{figure*}[htbp]
	\centering
		\includegraphics[width=0.95\textwidth]{images/workspace_newclass.png}
	\caption{Workbench with HelloWorld.java}
	\label{fig:workspace_newclass}
\end{figure*}


\subsection{Adding a New Aspect to Your Project}
Create a new Class and name it \code{World}. Edit the buffer so it looks like listing \ref{lst:aspect} and then save it:
\begin{lstlisting}[basicstyle=\small\it,caption=An \caesarj -cclass including an aspect,label=lst:aspect,name=listing:aspect,frame=none]{}
package myPackage;

public deployed cclass World {
	
	pointcut p(HelloWorld c) : execution(void HelloWorld.sayHelloTest(String)) && this(c);
   
	after(HelloWorld d) : p(d)
	{
		System.out.println("After Hello World");
	} 
} 
\end{lstlisting}

Furthermore you will need a "'Main-Class"' to run the project. Just create one like this:

\begin{lstlisting}[basicstyle=\small\it,caption=An \caesarj -java-class including an main method,label=lst:main,name=listing:main,frame=none]{}
package myPackage;

public class MAIN {
	public static void main(String[] args) {
		HelloWorld test = new HelloWorld();
		test.sayHelloTest("Hello World");
	}
}
\end{lstlisting}

Make a clean Build of the project, and the outline view populates like in figure \ref{fig:aspect2}. Expand the \code{after()} node.

\begin{figure*}[htbp]
	\centering
		\includegraphics[width=0.35\textwidth]{images/aspect2.png}
	\caption{Outline view with content}
	\label{fig:aspect2}
\end{figure*}

You can see that this advice is affecting the \code{HelloWorld.sayHello()} method. Clicking on the \code{HelloWorld.sayHello()} node in the outline takes you to the declaration of \code{HelloWorld.sayHello()}.\\
Notice the \textit{advice annotation} in the editor buffer (highlighted) and that the \code{sayHello} method in the outline view shows that it is advised by the \textit{World aspect}. It should look like in figure \ref{fig:aspect3}.

\begin{figure*}[htbp]
	\centering
		\includegraphics[width=1.0\textwidth]{images/aspect3.png}
	\caption{Advice relationship}
	\label{fig:aspect3}
\end{figure*}

Selecting the \code{World.after()} node in the outline view takes you back to the advice declaration. Right-clicking on the advice annotation brings up a context menu that also allows you to navigate to the advice.



\subsection{Running an Caesar Program}
Select your Caesar project in the Package Explorer. Drop-down the \markedtext{Run} icon on the toolbar and click \markedtext{Run...}\\
Select \markedtext{Java Application} in the left-hand tab and click \markedtext{New}.
Name this configuration \code{HelloWorld} and then click \markedtext{Search} to find the main class. Select \code{HelloWorld} as described in figure \ref{fig:run}.

\begin{figure*}[htbp]
	\centering
		\includegraphics[width=0.80\textwidth]{images/run.png}
	\caption{Running a \caesarj ~program}
	\label{fig:run}
\end{figure*}

Click \markedtext{Apply} and then \markedtext{Run}.\\
You should see the output of the \code{HelloWorld} ~class and the \code{World} ~aspect in the console like shown in figure \ref{fig:console}.

\begin{figure*}[htbp]
	\centering
		\includegraphics[width=0.80\textwidth]{images/run.png}
	\caption{Programs output}
	\label{fig:console}
\end{figure*}
To run this configuration again, just click on the \markedtext{Run} icon placed on the toolbar.


\subsection{Debugging Caesar Programs}
You can debug Caesar programs using the normal Java debugger. To set a breakpoint, right-click in the gutter of the editor and choose "Toggle Breakpoint", or simply double-click in the gutter.\\\\
\includegraphics[width=0.90\textwidth]{images/brake_point.png}\\

With one or more breakpoints set, you launch the Eclipse debugger in the normal way by clicking on the debug icon in the toolbar.\\\\
\includegraphics[width=0.80\textwidth]{images/debug1.png}\newpage

You can use the Java Debug step filters (Window -$>$ Preferences -$>$ Java -$>$ Debug -$>$ Step Filtering) to make this process a little easier.
Note: A current limitation is that you cannot step into advices.

\newpage
\section{Propertie and Shortcuts}
If you have opened the Caesar Perspective, there are some configurations left. Open \markedtext{Window} $\rightarrow$ \markedtext{Customise Perspective}. Check the \markedtext{Caesar} checkbox as shown in figure \ref{fig:propert}.

\begin{figure*}[htbp]
	\centering
		\includegraphics[width=0.60\textwidth]{images/propert.png}
	\caption{Selection the \caesarj ~perspective}
	\label{fig:propert}
\end{figure*}

If this is done, two new Buttons will appear in the toolbar like in figure \ref{fig:toolbar}.

\begin{figure*}[htbp]
	\centering
		\includegraphics[width=1.0\textwidth]{images/toolbar.png}
	\caption{\caesarj ~toolbar shortcuts}
	\label{fig:toolbar}
\end{figure*}

Figure \ref{fig:properties} shows the \caesarj -Configuration-Wizard, which will be displayed by pressing the \markedtext{P}-Button.

\begin{figure*}[htbp]
	\centering
		\includegraphics[width=0.60\textwidth]{images/view_properties.png}
	\caption{\caesarj -Configuration-Wizard}
	\label{fig:properties}
\end{figure*}

The \markedtext{A}-Button toggles the "Annotation-While-Typing" option on or off. Even for the Java-Editor.\\
A main feature of the \cjdt ~is the automatic annotation toggling while switching between the \caesarj - and the Java-editor.




\newpage
\section{Using the Visualiser and views}
If this is the first time you use the \cjdt, switch to the \caesarj ~perspective by selecting \markedtext{Window} $\rightarrow$ \markedtext{Open Perspective} $\rightarrow$ \markedtext{Other}. Pick \markedtext{CaesarJDT Perspective} (see figure \ref{fig:select_persp}) in the list.

\begin{figure*}[htbp]
	\centering
		\includegraphics[width=0.35\textwidth]{images/select_persp.png}
	\caption{Perspective selection}
	\label{fig:select_persp}
\end{figure*}

This perspective extends the Java perspective. Especially a new view is available. The \markedtext{\caesarj ~Hierarchy View}. See section \ref{hierarchyview} for detailed information.\\
You can switch between the Java and Caesar Visualization perspectives using the perspective icons in the top right of the menu bar.\\
\subsection{Outline view}
The outline view is showing structural members and crosscutting relationships. It extends the Java outline view by additional information (e.g advice declarations to the places it advises). A sample outline view bar is shown in figure \ref{fig:outline_view}. \textbf{TODO Bild noch nicht das richtige.}\\

\begin{figure*}[htbp]
	\centering
		\includegraphics[width=0.35\textwidth]{images/outline.png}
	\caption{Outline View}
	\label{fig:outline_view}
\end{figure*}

\subsection{Hierarchy View\label{hierarchyview}}
% It displays the hierarchical relationships of \caesarj ~cclasses.
A \caesarj ~hierarchy view displays the hierarchical relationships of \caesarj ~cclasses. That means, that for each cclass their super-classes are displayed under the \markedtext{Super} node (see figure \ref{fig:hierarchy_view}). If the class contains nested classes (\markedtext{Contains} node) there are two displaying modes available for them:
\begin{itemize}
	\item[\textbf{Super:}] For each nested class their super classes are displayed.
	\item[\textbf{Sub:}] For each nested class their sub classes are displayed. If a sub class has two super classes the linearized inheritance relation is displayed in brackets after the class name.
\end{itemize}

\begin{figure*}[htbp]
	\centering
		\includegraphics[width=0.35\textwidth]{images/hierarchy.png}
	\caption{\caesarj ~hierarchy view}
	\label{fig:hierarchy_view}
\end{figure*}

The modes can be switched by pressing the control button in the upper-right of the view. The second part of the view, named "Mixin view", shows the mixin composition of the currently selected (nested-) cclass.\\
\textbf{Note:} Because this view needs meta information from the compiler, the view refreshes when a project was (re-)built successfully.



\end{document}