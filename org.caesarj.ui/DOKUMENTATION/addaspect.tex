\subsection{Adding a New Aspect to Your Project}
Create a new Class and name it \code{World}. Edit the buffer so it looks as follows and then save it:\\\\
\begin{lstlisting}[basicstyle=\small\it,caption=Parser endElement Event,label=lst:endelement,name=listing:endelement,frame=none]{}
package myPackage;
//TEST
public cclass World { 	
      after(HelloWorld c, String a) :
      	(call(void sayHello(String)) && this(c) && args(a))
		{
            System.out.println("Hello to you too...");
		}
}\end{lstlisting}
%\includegraphics[width=0.80\textwidth]{images/aspect.png}\\

Make a clean Build of the project, and the outline view populates. Expand the \code{after()} node.\\\\
\includegraphics[width=0.5\textwidth]{images/aspect2.png}\newpage

You can see that this advice is affecting the HelloWorld.sayHello() method. Clicking on the \code{HelloWorld.sayHello()} node in the outline takes you to the declaration of HelloWorld.sayHello().

Notice the \code{advice annotation} in the editor buffer (highlighted) and that the \code{sayHello} method in the outline view shows that it is advised by the World aspect.\\\\
\includegraphics[width=0.95\textwidth]{images/aspect3.png}\\\\

Selecting the \code{World.after()} node in the outline view takes you back to the advice declaration. Right-clicking on the advice annotation brings up a context menu that also allows you to navigate to the advice.\\\\
\textbf{TODO BILD FEHLT KONNTE ICH NICHT SCREENSHOOT MACHEN  Weil es nicht geht EDITOR mit ADVICE CONTEXT}

