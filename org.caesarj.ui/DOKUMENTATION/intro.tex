\section{Introduction}
This documentation discribes how to use the \caesarj -Eclipse Plugin for the new programming language \caesarj.
\subsection{What is \caesarj ?}
\caesar ~is a new aspect-oriented programming language that extends JAVA with the following functions:
\begin{itemize}
	\item aspect\footnote{More information about \caesarj ~and aspects read this  \href{http://www.st.informatik.tu-darmstadt.de/database/publications/data/aosd03.pdf?id=70}{Paper}} functionality with runtime deployment of aspects
	\item multible inheritance
	\item produces 100\% pure Java byte code
	\item \dots
\end{itemize}
For more detailed information please visite \href{http://caesarj.org/}{http://caesarj.org/}.

\subsection{Why is a plugin needed?}
\caesarj ~extends the Java source code. A pure JAVA-Editor (e.g. the \jdt -Eclipse Plugin) would not be able to handle this kind of source code. Also the integrated JAVA compiler would not work. So an IDE is needed, that extends the \jdt. Some features of the \caesarj -Plugin are:
\begin{itemize}
	\item integrated caesarj builder
	\item codehighlighting for \caesarj ~expressions
	\item visualisation of aspects
	\item visualisation of multible inheritance
	\item \dots
\end{itemize}
For detailed description please see Section \ref{features}.


