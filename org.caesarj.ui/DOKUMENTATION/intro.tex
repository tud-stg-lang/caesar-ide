\section{Introduction}
This documentation discribes how to use the \caesarj -Eclipse Plugin for the new programming language \caesarj.
\subsection{What is \caesarj ?}
\caesar ~is a new aspect-oriented programming language that extends JAVA with the following functions:
\begin{itemize}
	\item aspect\footnote{\href{http://www.st.informatik.tu-darmstadt.de/database/publications/data/aosd03.pdf?id=70}{This paper} contains more information about \caesarj ~and aspects.} functionality with runtime deployment of aspects
	\item multiple inheritance
	\item aspecture polymorphisms
	\item dynamic deployment
	\item virtual classes with propagating mixin composition
	\item \dots
\end{itemize}
For more detailed information please visite \href{http://caesarj.org/}{http://caesarj.org/}.

\subsection{Why is the use of the \caesarj ~ Plugin?}
\caesarj ~extends the Java source code. A pure JAVA-Editor (e.g. the \jdt -Eclipse Plugin) would not be able to handle this kind of source code. Furthermore the integrated JAVA compiler would not work. So an IDE is needed, that extends the \jdt. 
\subsection{What are the features of the \jdt}
The \jdt -Eclipse plugin supports many features for JAVA programmers. Some highlights are:
\begin{itemize}
	\item code highlighting
	\item live code annotation 
	\item outline visualisation
	\item type hierarchy
	\item calling hierarchy
	\item and many more
\end{itemize}
The \caesarj ~plugin trys to extend these features and add \caesarj ~specific features.

\subsection{Which extensions do we need for \caesarj}
To provide a good IDE for the \caesarj ~programming language, we needed to extend the \jdt ~plugin with the following features:

\begin{itemize}
	\item syntax-highlighting for \caesarj ~expressions
	\item crosscutting views
	\item integrated \caesarj ~builder
	\item new outline bar (with visualisation of aspects)
	\item new side ruler (with visualisation of aspects)
	\item \caesarj ~hierarchy view (visualisation of multiple inheritance)
\end{itemize}
For detailed description please see Section \ref{features}.


