\section{Introduction}
This documentation describes how to use the \caesarj -Eclipse Plugin.


\subsection{What is \caesarj ?}
\caesar ~is a new aspect-oriented programming language, which addresses the most important goals of the software design: modularity, reuse, flexibility and correctness. It is easy to learn because it fully integrates with the Java programming language. All new language extensions are compiled to efficient Java byte-code. The \caesar ~highlights are:
\begin{itemize}
	\item Virtual Classes
	\item Mixin Composition
	\item Collaboration Interfaces	
	\item Bindings
	\item Aspectual Polymorphism	
	\item Dynamic Deployment		
\end{itemize}

For more detailed information please visit \href{http://caesarj.org/}{http://caesarj.org/}.

\subsection{About the \caesarj ~ Plugin}
\caesarj ~extends the Java language with new syntax and semantics. In order to support the daily development with \caesar ~we have developed the Caesar Development Tools (CDT) consisting of a plugin for the Eclipse Integrated Development Environment (IDE). The Eclipse IDE has already built-in support for Java development. Some highlights of Eclipse's \jdt ~are:
\begin{itemize}
	\item code highlighting
	\item live code annotation 
	\item outline visualisation
	\item type hierarchy
	\item calling hierarchy
\end{itemize}

In order to provide a good IDE support for the \caesarj ~programming language, we have extended the \jdt ~plugin with the following features:

\begin{itemize}
	\item syntax-highlighting for \caesarj ~code
	\item crosscutting views
	\item integrated \caesarj ~builder
	\item new outline bar (with visualisation of aspects)
	\item new side ruler (with visualisation of aspects)
	\item \caesarj ~hierarchy view (visualisation of multiple inheritance)
\end{itemize}

For detailed description please see Section \ref{features}.
